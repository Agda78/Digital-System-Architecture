\documentclass[../main.tex]{subfiles}

\begin{document}

\chapter{Reti combinatorie elementari}

\section*{Esercizio 1: Multiplexer 16:1}
\addcontentsline{toc}{section}{Esercizio 1: Multiplexer 16:1}

\subsection*{Esercizio 1.1}
\addcontentsline{toc}{subsection}{Esercizio 1.1}
Progettare, implementare in VHDL e testare mediante simulazione un multiplexer indirizzabile 16:1, utilizzando un approccio di progettazione per composizione a partire da multiplexer 4:1.

\subsubsection*{Progetto e architettura}
\addcontentsline{toc}{subsubsection}{Progetto e architettura}
\textit{Descrizione dell’approccio di progetto utilizzato, disegno architetturale del sistema e dei suoi componenti, descrizione delle funzionalità.}

\subsubsection*{Implementazione}
\addcontentsline{toc}{subsubsection}{Implementazione}
\textit{Codice VHDL dei componenti significativi: componenti elementari riutilizzati in diversi progetti vanno inseriti in un’appendice.}

\subsubsection*{Simulazione}
\addcontentsline{toc}{subsubsection}{Simulazione}
\textit{Descrizione dei testbench di maggiore rilevanza utilizzati per testare il sistema e i suoi componenti e discussione dei principali risultati in simulazione.}

\subsubsection*{Sintesi su board di sviluppo}
\addcontentsline{toc}{subsubsection}{Sintesi su board di sviluppo}
\textit{Se richiesto: descrizione dell’architettura complessiva necessaria per la sintesi su board di sviluppo (nel caso ci siano eventuali componenti aggiuntivi per la gestione dell’I/O); file di constraint utilizzato per il progetto.}

\subsubsection*{Timing analysis}
\addcontentsline{toc}{subsubsection}{Timing analysis}
\textit{Se richiesto: discussione dei risultati della timing analysis sui circuiti realizzati.}

\end{document}
